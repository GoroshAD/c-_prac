\documentclass{article}
\usepackage{graphicx} % Required for inserting images
\usepackage[T1,T2A]{fontenc}     % форматы шрифтов
\usepackage[utf8x]{inputenc}     % кодировка символов, используемая в данном файле
\usepackage[russian]{babel}      % пакет русификации
\usepackage{tikz}                % для создания иллюстраций
\usepackage{pgfplots}            % для вывода графиков функций
\usepackage{geometry}		 % для настройки размера полей
\usepackage{indentfirst}         % для отступа в первом абзаце секции
\usepackage{pb-diagram}
\usepackage{amsmath}
\usepackage{amsfonts}

\geometry{a4paper,left=30mm,top=30mm,bottom=30mm,right=30mm}

\title{Математическая постановка задачи оптимизации
расписания с использованием алгоритма имитации отжига}
\author{Артём Горошко (Вариант II)}

\begin{document}

\maketitle

\section{Постановка задачи}
Дано N независимых работ, для каждой работы задано время выполнения. Требуется построить расписание выполнения работ без прерываний на M процессорах. На расписании должно достигаться минимальное значение критерия, где критерием является суммарное время ожидания (т.е. сумма, по всем работам в расписании, времён завершения работ).
\section{Математическая постановка задачи}
\subsection{Дано}
\begin{enumerate}
    \item Множество P = $\{p_1, p_2, ..., p_N\}$ - множество работ. $|P| = N$ - всего работ.
    \item Функция $t: P \rightarrow \mathbb{N}_0$ - функция определения времени работы.
    \item Множество M = $\{m_1, m_2, ..., m_K\}$ - множество процессоров. $|M| = K$ - всего процессоров.
\end{enumerate}

\subsection{Необходимо найти}
\begin{enumerate}
    \item Множество S = $\{s_1, s_2, ..., s_N\}$ - множество, в котором каждый элемент имеет следующий вид: $$s_i = (m_j, t_i)$$ где i - номер выполняемой работы, j - номер процессора, на котором происходит выполнение работы i, $t_i$ - время начала выполнения работы i. Время окончания работы определяется формулой $$t_i^{finish} = t_i + t(p_i)$$ 
    Ограничение на множество расписаний:
    \begin{itemize}
        \item $t_i^{finish} = t_{k}$ где $s_i = (m_j, t_i), s_k = (m_j, t_k)$ и работа k начинает выполняться сразу после работы i. То есть, иными словами, время между концом одной работы и началом следующей работы на одном и том же процессоре равно 0.
        \item $t_i^{finish} - t_i = t(p_i)$ - то есть прерываний при выполнении работы нет.
    \end{itemize}
    Это множество должно удовлетворять следующему критерию: $$min(\sum^N_it_i + t(p_i))$$ Таким образом, должен достигаться минимум суммарного времени завершения всех работ.
\end{enumerate}

\end{document}
